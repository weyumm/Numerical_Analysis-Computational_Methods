\documentclass[UTF8]{ctexart}
\usepackage{tcolorbox}
\usepackage{mdframed}
\usepackage{amsmath}  
\usepackage{amssymb}  
\usepackage{graphicx}
\usepackage{float}  
\usepackage{wrapfig}  
\usepackage{algorithm}  
\usepackage{algorithmic} 
\usepackage{hyperref}
\usepackage{listings}
\usepackage{color,xcolor} 
\usepackage{colortbl}
\usepackage{graphicx}
\usepackage{booktabs} %绘制表格
\usepackage{caption2} %标题居中
\usepackage{geometry}
\usepackage{array}
\usepackage{subfigure} 
\usepackage{longtable}
\usepackage{multirow}
\usepackage{enumerate}
\usepackage{booktabs} % 用于绘制三线表
\usepackage{siunitx} % 用于处理单位和数字
\usepackage{threeparttable} % 用于添加表格注释
\usepackage{rotating} % 用于旋转表格
\pagestyle{plain} %页眉消失
\lstset{
	language=Matlab,
	basicstyle=\ttfamily\small,
	keywordstyle=\color{blue},
	commentstyle=\color{green!60!black},
	stringstyle=\color{red},
	numbers=left,
	numberstyle=\tiny\color{gray},
	stepnumber=1,
	numbersep=5pt,
	frame=single,
	breaklines=true,
	tabsize=4,
	captionpos=b,
	showstringspaces=false
}
\lstset{
	language=Python,
	basicstyle=\ttfamily\small,
	keywordstyle=\color{blue},
	commentstyle=\color{gray},
	stringstyle=\color{red},
	numbers=left,
	numberstyle=\tiny\color{gray},
	stepnumber=1,
	numbersep=5pt,
	frame=single,
	breaklines=true,
	tabsize=4,
	captionpos=b,
	showstringspaces=false,
	escapeinside=``
}
\lstdefinestyle{cppstyle}{
	language=C++,
	basicstyle=\ttfamily\small,
	keywordstyle=\color{blue},
	commentstyle=\color{gray},
	stringstyle=\color{red},
	numberstyle=\tiny\color{gray},
	numbers=left,
	stepnumber=1,
	numbersep=5pt,
	frame=single,
	breaklines=true,
	tabsize=4,
	showstringspaces=false,
	escapeinside=``
}
\geometry{a4paper,left=2.5cm,right=2.5cm,top=2.5cm,bottom=2.5cm}%设置页面尺寸
\lstset{
	numbers=left, %设置行号位置
	numberstyle=\tiny, %设置行号大小
	keywordstyle=\color{blue}, %设置关键字颜色
	commentstyle=\color[cmyk]{1,0,1,0}, %设置注释颜色
	escapeinside=``, %逃逸字符(1左面的键),用于显示中文
	breaklines, %自动折行
	extendedchars=false, %解决代码跨页时,章节标题,页眉等汉字不显示的问题
	xleftmargin=1em,xrightmargin=1em, aboveskip=1em, %设置边距
	tabsize=4, %设置tab空格数
	showspaces=false %不显示空格
}

\definecolor{lightblue}{rgb}{0.87, 0.92, 1.0}
\definecolor{lightred}{rgb}{1.0, 0.8, 0.8}
\definecolor{lightyellow}{rgb}{1.0, 0.95, 0.8}
\hypersetup{
	colorlinks=true,
	colorlinks=true,
	linkcolor=black,
	urlcolor=blue,
	citecolor=red,
	filecolor=green,
	pdfborder={0 0 0}
}
\usepackage{pdfpages}  

\usepackage{eso-pic}
\usepackage{fancyhdr}
\pagestyle{fancy}
\fancyhf{} % 清除当前页眉页脚的设置
\fancyhead[R]{\thepage} % 右上角页眉
\fancyhead[L]{计算方法作业} % 左上角页眉

\title{\textbf{Chapter1-误差与有效数字}}
\author{姓名:张恒祯 \, 学号:2351071 \,}
\date{日期:2025 年3月2日}

\begin{document}
	\maketitle
\section{思路图}
\begin{figure}[htbp]
	\centering
	\includegraphics[width=1.0\linewidth]{figure/ch1diagram}
	\label{fig:ch1diagram}
\end{figure}

	\section{必做题(奇数)}	
\begin{tcolorbox}[colback=blue!5!white,colframe=blue!75!black,title=1 近似值误差分析]
	\setcounter{equation}{0}
	\textbf{解题诀窍:取最后一位乘$\frac{1}{2}$。}
	\textbf{(1) $x_1^* = 0.024$}
	\begin{equation}
	\delta_{x_1} = 0.0005,\quad \varepsilon_{x_1} = \frac{0.0005}{0.024} \approx 2.08\%,\quad \text{2位有效数字}
	\end{equation}
	
	\textbf{(2) $x_2^* = 0.4135$}
	\begin{equation}
	\delta_{x_2} = 0.00005,\quad \varepsilon_{x_2} = \frac{0.00005}{0.4135} \approx 0.012\%,\quad \text{4位有效数字}
	\end{equation}
	
	\textbf{(3) $x_3^* = 57.50$}
	\begin{equation}
	\delta_{x_3} = 0.005,\quad \varepsilon_{x_3} = \frac{0.005}{57.50} \approx 0.0087\%,\quad \text{4位有效数字}
	\end{equation}
	
	\textbf{(4) $x_4^* = 60000$}
	\begin{equation}
	\delta_{x_4} = 0.5,\quad \varepsilon_{x_4} = \frac{0.5}{60000} \approx 0.00\%,\quad \text{5位有效数字}
	\end{equation}
	
	\textbf{(5) $x_5^* = 8 \times 10^5$}
	\begin{equation}
	\delta_{x_5} = 5 \times 10^4,\quad \varepsilon_{x_5} = \frac{5 \times 10^4}{8 \times 10^5} = 6.25\%,\quad \text{1位有效数字}
	\end{equation}
\end{tcolorbox}

\begin{tcolorbox}[colback=blue!5!white,colframe=blue!75!black,title=3 $\sqrt{11}$的有效数字分析]
	设需要$n$位有效数字,相对误差限满足:
	\begin{equation}
	\frac{1}{2 \times 10^{n-1}} \leq 0.001
	\end{equation}
	
	取$\sqrt{11} \approx 3.3166$,首位数字$a_1=3$,修正公式:
	\begin{equation}
	\frac{1}{2(a_1+1) \times 10^{n-1}} \leq 0.001 \Rightarrow n \geq 3.70
	\end{equation}
	
	向上取整得:\boxed{4}位有效数字
\end{tcolorbox}

\begin{tcolorbox}[colback=blue!5!white,colframe=blue!75!black,title=5 边长误差限计算]
	\textbf{已知:}正方形边长$l \approx100\ \mathrm{cm}$,要求面积误差$\Delta A \leq1\ \mathrm{cm}^2$。
	
	\textbf{步骤:}
	面积公式$A=l^2$,误差传播关系:
	\begin{equation}
	\Delta A \approx 2l \cdot \Delta l
	\end{equation}
	
	解边长误差限:
	\begin{equation}
	\Delta l \leq \frac{\Delta A}{2l} = \frac{1}{2\times100} = 0.005\ \mathrm{cm}
	\end{equation}
	
	单位换算:
	\begin{equation}
	0.005\ \mathrm{cm} = 0.05\ \mathrm{mm}
	\end{equation}
	
	结论:边长测量允许最大误差为\boxed{\pm0.005\ \mathrm{cm}}
\end{tcolorbox}
% ======================= 问题7解答 =======================
\begin{tcolorbox}[colback=blue!5!white,colframe=blue!75!black,title=问题7 方程求根]
	\textbf{方程:} $ x^2 - 40x + 1 = 0 $,已知 $ \sqrt{399} \approx 19.975 $
	
	\textbf{解法:}
	判别式计算:
	$$
	\Delta = 40^2 - 4 \times 1 = 1596 = 4 \times 399 \quad \Rightarrow \quad \sqrt{\Delta} = 2\sqrt{399} \approx 39.95
	$$
	
	\textbf{根1(避免相近数相减):}
	$$
	x_1 = \frac{40 + 39.95}{2} = \frac{79.95}{2} = 39.975 \quad (\text{直接计算})
	$$
	
	\textbf{根2(利用韦达定理):}
	$$
	x_2 = \frac{1}{x_1} = \frac{1}{39.975} = 0.02501876 \approx 0.025019
	$$
	
	结果验证:
	$$
	x_1 \times x_2 = 39.975 \times 0.025019 \approx 1.0000 \quad (\text{满足方程})
	$$
	
	结论:根为 \boxed{39.98} 和 \boxed{0.02502}(四位有效数字)。
\end{tcolorbox}
% ======================= 问题9解答 =======================
\begin{tcolorbox}[colback=blue!5!white,colframe=blue!75!black,title=问题9 优化计算方法]
	\textbf{(1) $ 1 - \cos1^\circ $:}
	采用泰勒展开避免相近数相减:
	$$
	1 - \cos x \approx \frac{x^2}{2} - \frac{x^4}{24},\quad x = \frac{\pi}{180} \approx 0.017453
	$$
	计算结果:
	$$
	1 - \cos1^\circ \approx 1.523 \times 10^{-4} \quad (\text{四位有效数字})
	$$
	
	\textbf{(2) $ \ln(30 - \sqrt{30^2 - \sqrt{30^3 -1}}) $:}
	分层计算保留中间量高精度:
	$$
	\begin{aligned}
	&\sqrt{30^3 -1} \approx 164.3160 \quad (\text{六位开方}) \\
	&\sqrt{30^2 - 164.3160} = \sqrt{735.684} \approx 27.1190 \\
	&\ln(30 - 27.1190) = \ln(2.8810) \approx 1.0575
	\end{aligned}
	$$
	
	\textbf{(3) $ \frac{1 - \cos x}{\sin x} $($ |x| $ 极小):}
	三角恒等变换:
	$$
	\frac{1 - \cos x}{\sin x} = \tan\frac{x}{2} \approx \frac{x}{2} \quad (|x| \to 0)
	$$
	
	\textbf{(4) $ \frac{1}{N+1} $($ N $ 充分大):}
	泰勒展开保留主项:
	$$
	\frac{1}{N+1} = \frac{1}{N} - \frac{1}{N^2} + o\left(\frac{1}{N^2}\right)
	$$
	
	结论:通过数学变换和分层计算可提高精度。
\end{tcolorbox}

	\section{选做题(偶数)}
\begin{tcolorbox}[colback=green!5!white,colframe=green!75!black,title=2 $\pi^{10}$的有效数字分析]
	设需要$n$位有效数字,相对误差限满足:
	\begin{equation}
	\frac{1}{2 \times 10^{n-1}} \leq 0.001
	\end{equation}
	
	取$\pi^{10} \approx 93648.05$,首位数字$a_1=9$,修正公式:
	\begin{equation}
	\frac{1}{2(a_1+1) \times 10^{n-1}} \leq 0.001 \Rightarrow n \geq 4.47
	\end{equation}
	
	向上取整得:\boxed{5}位有效数字
\end{tcolorbox}
\begin{tcolorbox}[colback=green!5!white,colframe=green!75!black,title=4 纬度计算误差分析]
\textbf{已知:}纬度$\phi = 45^\circ 0'2''$(测量到秒),求$\sin\phi$的误差。

\textbf{步骤:}
角度测量误差$\Delta\phi = \pm0.5''$,转换为弧度:
\begin{equation}
\Delta\phi = 0.5'' \times \frac{1^\circ}{3600''} \times \frac{\pi}{180} = 2.424\times10^{-6}\ \mathrm{rad}
\end{equation}

利用微分近似计算$\sin\phi$误差:
\begin{equation}
\Delta(\sin\phi) \approx |\cos\phi| \cdot \Delta\phi
\end{equation}

代入$\phi=45^\circ$,$\cos45^\circ=\frac{\sqrt{2}}{2}$:
\begin{equation}
\Delta(\sin\phi) \approx 0.7071 \times 2.424\times10^{-6} = 1.715\times10^{-6}
\end{equation}

结论:$\sin\phi$的误差为$\pm1.7\times10^{-6}$
\end{tcolorbox}

\begin{tcolorbox}[colback=green!5!white,colframe=green!75!black,title=6 多项式误差传播分析]
	\textbf{已知:}$y=P(x)=x^2+x-1150$,精确值$x=\frac{100}{3}$,近似值$x^*=33$。
	
	\textbf{步骤:}
	\begin{equation}
	y_{\text{精确}} = P\left(\frac{100}{3}\right) = \left(\frac{100}{3}\right)^2 + \frac{100}{3} -1150 = -\frac{50}{9} \approx-5.5556
	\end{equation}
	
	\begin{equation}
	y^* = P(33) = 33^2 +33 -1150 = -28
	\end{equation}
	
	相对误差计算:
	\begin{equation}
	\varepsilon_x = \frac{|33-\frac{100}{3}|}{\frac{100}{3}} = \frac{\frac{1}{3}}{\frac{100}{3}} = 1\%
	\end{equation}
	
	\begin{equation}
	\varepsilon_y = \frac{|-28-(-\frac{50}{9})|}{\left|\frac{50}{9}\right|} = \frac{\frac{202}{9}}{\frac{50}{9}} = 404\%
	\end{equation}
	
	结论:相对误差为$\varepsilon_x=\boxed{1\%}$,$\varepsilon_y=\boxed{404\%}$
\end{tcolorbox}
% ======================= 问题8解答 =======================
\begin{tcolorbox}[colback=green!5!white,colframe=green!75!black,title=问题8 误差传播分析]
	\textbf{表达式:}
	$$
	(1)\ y_1 = \frac{x^2 \pm \varepsilon}{100},\quad (2)\ y_2 = \frac{x^2 \pm \varepsilon}{0.001}
	$$
	
	\textbf{误差分析:}
	对 $ y = \frac{f(x)}{a} $,绝对误差满足 $ \Delta y = \frac{\Delta f}{|a|} $
	
	\textbf{(1) $ y_1 $ 计算:}
	$$
	\Delta y_1 = \frac{|2x \varepsilon| + \varepsilon}{100} \leq \frac{(2|x| +1)\varepsilon}{100}
	$$
	特性:分母100使绝对误差缩小100倍,相对误差保持不变。
	
	\textbf{(2) $ y_2 $ 计算:}
	$$
	\Delta y_2 = \frac{|2x \varepsilon| + \varepsilon}{0.001} = 1000(2|x| +1)\varepsilon
	$$
	特性:分母0.001使绝对误差放大1000倍,相对误差仍不变。
	
	结论:表达式(1)误差缩小,表达式(2)误差放大。
\end{tcolorbox}
% ======================= 第10题解答 =======================
\begin{tcolorbox}[colback=green!5!white,colframe=green!75!black,title=10. 数值稳定性公式选择]
	
	% ---------- (1) |x| ≤ 1 ----------
	\textbf{(1) 已知 $|x| \leq 1$}
	\begin{align*}
	\text{(a)}\ & y = \frac{1}{1+2x} - \frac{1-x}{1+x} \\
	\text{(b)}\ & y = \frac{2x^2}{(1+2x)(1+x)}
	\end{align*}
	
	\textbf{选择:} \boxed{(b)}
	\textbf{原因:}
	当 $|x| \ll 1$ 时,(a) 式中两分数项的值近似相等($\approx 1 - x$),直接相减会导致有效数字丢失。例如 $x=0.001$ 时:
	$$
	\frac{1}{1.002} - \frac{0.999}{1.001} \approx 0.9980 - 0.9980 = 0\ (\text{实际应有}\ y \approx 2 \times 10^{-6})
	$$
	而 (b) 式通过通分消除减法操作,计算更稳定。
	
	% ---------- (2) |x| ≫ 1 ----------
	\textbf{(2) 已知 $|x| \gg 1$}
	\begin{align*}
	\text{(a)}\ & y = \frac{2}{x \left( \sqrt{x + \frac{1}{x}} + \sqrt{x - \frac{1}{x}} \right)} \\
	\text{(b)}\ & y = \sqrt{x + \frac{1}{x}} - \sqrt{x - \frac{1}{x}}
	\end{align*}
	
	\textbf{选择:} \boxed{(a)}
	\textbf{原因:}
	当 $|x| \gg 1$ 时,(b) 式中两个平方根项近似相等,相减导致有效数字丢失。例如 $x=1000$ 时:
	$$
	\sqrt{1000.001} - \sqrt{999.999} \approx 31.6228 - 31.6222 = 0.0006\ (\text{实际应有}\ y \approx 3.16 \times 10^{-8})
	$$
	而 (a) 式通过分母相加保留修正项,避免减法误差。
	
	% ---------- (3) |x| ≤ 1 ----------
	\textbf{(3) 已知 $|x| \leq 1$}
	\begin{align*}
	\text{(a)}\ & y = \frac{2\sin^2 x}{x} \\
	\text{(b)}\ & y = \frac{1 - \cos 2x}{x}
	\end{align*}
	
	\textbf{选择:} \boxed{(a)}
	\textbf{原因:}
	当 $|x| \ll 1$ 时,(b) 式中 $1 - \cos 2x \approx 2x^2$ 为相近数相减,例如 $x=0.001$ 时:
	$$
	\frac{1 - 0.999998}{0.001} = \frac{2 \times 10^{-6}}{0.001} = 0.002\ (\text{有效数字丢失})
	$$
	而 (a) 式通过 $\sin^2 x = x^2 - \frac{x^4}{3} + \cdots$ 直接计算更稳定。
	
	% ---------- (4) p ≫ q ----------
	\textbf{(4) 已知 $p \gg q > 0$}
	\begin{align*}
	\text{(a)}\ & y = \frac{q^2}{p + \sqrt{p^2 + q^2}} \\
	\text{(b)}\ & y = \sqrt{p^2 + q^2} - p
	\end{align*}
	
	\textbf{选择:} \boxed{(a)}
	\textbf{原因:}
	当 $p \gg q$ 时,(b) 式中 $\sqrt{p^2 + q^2} \approx p + \frac{q^2}{2p}$,相减后:
	$$
	y \approx \frac{q^2}{2p}\ (\text{但计算时因大数相减丢失精度})
	$$
	而 (a) 式分母 $p + \sqrt{p^2 + q^2} \approx 2p$,直接得到:
	$$
	y = \frac{q^2}{2p}\ (\text{无减法操作,精度更高})
	$$
	
\end{tcolorbox}


\section{有效数字计算函数代码}
\begin{lstlisting}[caption={MATLAB 有效数字计算函数 (getdigits.m)}, label=lst:getdigits]
function [n, e] = getdigits(xtrue, x)
% GETDIGITS Calculate significant digits and error bound
% Inputs:
%   xtrue - True value
%   x     - Approximate value
% Outputs:
%   n     - Number of significant digits
%   e     - Absolute error bound

err = abs(xtrue - x);  % Calculate absolute error
[x_norm, m] = scientific_notation(x);  % Convert to scientific notation
[err_norm, q] = scientific_notation(err);  % Convert error

% Determine significant digits based on leading error digit
if err_norm < 5
n = m - q;
else
n = m - q - 1;
end

% Calculate error bound (0.5 * 10^(m-n))
e = 0.5 * 10^(m - n);

% Nested function: Scientific notation conversion
function [x_scaled, exponent] = scientific_notation(x)
x = abs(x);
exponent = 0;  % Initialize exponent

if x == 0
x_scaled = 0;
exponent = 0;
return;
end

% Normalize to 1 ≤ x_scaled < 10
while x >= 10
x = x / 10;
exponent = exponent + 1;
end
while x < 1
x = x * 10;
exponent = exponent - 1;
end

x_scaled = x;
exponent = exponent;  % Final exponent
end
end
\end{lstlisting}



	\section{附作业手写记录照片}
	\begin{figure}[htbp]
		\centering
		\includegraphics[width=0.9\linewidth]{figure/homework}
		\label{fig:homework}
	\end{figure}
\section{Python 有效数字计算函数}

\begin{lstlisting}[caption={Python 有效数字计算函数 (getdigits.py)}, label=lst:getdigits-python]
from typing import Tuple
import math

def getdigits(xtrue: float, x: float) -> Tuple[int, float]:
"""
计算近似值的有效数字位数及误差限

Args:
xtrue (float): 真实值
x (float): 近似值

Returns:
Tuple[int, float]: (有效数字位数, 绝对误差限)
"""

def scientific_notation(num: float) -> Tuple[float, int]:
"""
将数值转换为标准科学计数法形式 (1 ≤ mantissa < 10)

Args:
num (float): 输入数值

Returns:
Tuple[float, int]: (尾数, 指数)
"""
if num == 0:
return 0.0, 0

num = abs(num)
exponent = 0

# 处理大数
while num >= 10:
num /= 10
exponent += 1

# 处理小数
while num < 1:
num *= 10
exponent -= 1

return num, exponent

# 计算绝对误差
err = abs(xtrue - x)

# 转换为科学计数法
x_scaled, m = scientific_notation(x)
err_scaled, q = scientific_notation(err) if err != 0 else (0.0, 0)

# 获取误差首位数字
first_err_digit = int(math.floor(err_scaled)) if err_scaled >= 1 else 0

# 判断有效位数
if first_err_digit < 5:
n = m - q
else:
n = m - q - 1

# 计算绝对误差限
e = 0.5 * (10 ​**​ (m - n))

return n, e


# 测试用例
if __name__ == "__main__":
# 示例1: 整数情况
print(getdigits(123.45, 123.40))   # (4, 0.05)

# 示例2: 小数情况
print(getdigits(0.0245, 0.024))   # (2, 0.0005)

# 示例3: 零值处理
print(getdigits(0.0, 0.0))        # (0, 0.0)

# 示例4: 大数情况
print(getdigits(123456, 123400))  # (4, 50.0)
\end{lstlisting}

\section{C++ 有效数字计算函数}

\begin{lstlisting}[style=cppstyle,caption={C++ 有效数字计算函数 (getdigits.cpp)},label=lst:getdigits-cpp]
#include <iostream>
#include <cmath>
#include <utility>

using namespace std;

pair<int, double> getdigits(double xtrue, double x) {
/**
* 计算近似值的有效数字位数及误差限
* @param xtrue 真实值
* @param x 近似值
* @return pair<有效数字位数, 绝对误差限>
*/

auto scientific_notation = [](double num) -> pair<double, int> {
/**
* 将数值转换为标准科学计数法形式 (1 ≤ mantissa < 10)
* @param num 输入数值
* @return pair<尾数, 指数>
*/
if (num == 0) return {0.0, 0};

num = abs(num);
int exponent = 0;

// 处理大数
while (num >= 10) {
num /= 10;
exponent++;
}

// 处理小数
while (num < 1) {
num *= 10;
exponent--;
}

return {num, exponent};
};

// 计算绝对误差
double err = abs(xtrue - x);

// 转换为科学计数法
auto [x_scaled, m] = scientific_notation(x);
auto [err_scaled, q] = (err != 0) ? 
scientific_notation(err) : make_pair(0.0, 0);

// 获取误差首位数字
int first_err_digit = (err_scaled >= 1) ? 
floor(err_scaled) : 0;

// 判断有效位数
int n;
if (first_err_digit < 5) {
n = m - q;
} else {
n = m - q - 1;
}

// 计算绝对误差限
double e = 0.5 * pow(10, m - n);

return {n, e};
}

int main() {
// 示例测试
auto test1 = getdigits(123.45, 123.40);
cout << "(" << test1.first << ", " << test1.second << ")\n";  // (4, 0.05)

auto test2 = getdigits(0.0245, 0.024);
cout << "(" << test2.first << ", " << test2.second << ")\n"; // (2, 0.0005)

auto test3 = getdigits(0.0, 0.0);
cout << "(" << test3.first << ", " << test3.second << ")\n";  // (0, 0.0)

auto test4 = getdigits(123456, 123400);
cout << "(" << test4.first << ", " << test4.second << ")\n"; // (4, 50.0)

return 0;
}
\end{lstlisting}

\end{document}